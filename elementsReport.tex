\documentclass[12pt]{article}
\usepackage[margin=15mm]{geometry}
\usepackage{multicol}
\usepackage{enumitem}
\usepackage{listings}
\usepackage{fixltx2e}
\newcommand\tab[1][1cm]{\hspace*{#1}}
\setlength\itemsep{1em}
\lstset { 
	tabsize=2, 
}

\title{Elements Report Self}
\author{Kyle Dymowski, Aidan Miller}
\date{March 28, 2016}

\begin{document}
\maketitle

\section{The Team}
\paragraph{•} Two peope are accredited with the development of Self, David Ungar and Randy Smith, they began working on the language in 1986 while working at Xerox PARC then in 1987 they moved to Stanford University to build the compiler. 
\subsection{David Ungar}
\paragraph{} Dr. Ungar puersued his Ph.D in Computer Science at the University of California Berkeley from 1980 to 1985. Where he participated in architectural design of SOAR, Smalltalk on a RISC microprocessor. He also modified a simulator and virtual machine to benchmark architectural ideas, and designed, anaylised, implemented and measured the first two generation garbage collector. In 1985 he joined Stanford University as a Assistant Professor where he led the Self team, invented mirror-based reflection architecture and a stroage system for prototype-based languages. He then began working at Sun Microsystem Lavoratories in 1991 where he continued working on self until 1995. He left sun enterprises in 2006 and started working at IBM where he currently works on the Watson project. (https://www.linkedin.com/in/davidungar) 
\subsection{Randy Smith}
\paragraph{} Dr. Smith started his education with a double major in physics and mathmatics with the highest honor of summa cum laude in 1974 at the University California, Davis. He then pursued his PhD in Theoretical Physics graduating in 1981 from the University of California at San Diego. He began working on self in 1986 at Xerox PARC, then continued his work at Stanford in 1987, and finally finished the project at Sun Microsystems Laboritory in 1991. Where he still works today as a consulting member of the technical staff, he is currently working in the areas of modeling, simulation and optimization. 

\end{document}